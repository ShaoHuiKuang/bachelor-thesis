% !Mode:: "TeX:UTF-8"
\chapter{绪论}

\section{研究背景}

\quad 截至2016年3月,全球最大社交网络平台Facebook活跃用户量已经突破15.9亿。中国最大的社交媒体微博用户也早在2013年突破了6亿用户,国际知名社交平台Twitter也在2016年突破了13亿的注册用户量。随着这些在线社交网络的迅猛发展以及移动智能电话的大规模普及,社交网络分析引起了越来越多的来自计算机、社会学、数学等领域的学者广泛关注。社交网络的原始定义\cite{刘军2004社会网络分析导论,ccaggarwalintrosocial}来自于社会学,表示社会角色以及其交互关系的集合。而社会角色可以定义为独立的个人,也可以定义为家庭、学校或者国家等社会群体。而社会角色之间的的联系,则可以是任何无形(如两个人之间的朋友关系)或者有形(如国与国之间的合作)的交互关系,这些关系完全都可以由研究问题的学者自己定义。因此,由多个点(即社会角色)以及表示各个点之间关系(即为交互关系)的边所构成的网络,即为社交网络。在我们所生活的世界中,社交网络无处不在,如Email网络、学术网络或手机电话联系网络。虽然互联网的发展,出现了许多的在线社交网络,如Facebook、Twitter、Weibo等等。这一系列的社交网络的兴起促进了海内外各个领域的学者对其的研究,而其研究结果又被用到广告营销、社会服务、公共安全等各种不同的领域。如图1.1即为一个来自Friender网站的一个社交网络实例。图中可以看到,每个人都是一个点,而每条边表示两个人之间为朋友关系,将它们整体结合起来就构成了一个社交网络。

\quad 在以前的研究中,研究人员主要以在线社交网络为主要研究对象。但在移动互联网出现以前,用户只能通过Web页面登录到相应的社交网站。因此,以前的大多数研究都将重点放在了社交层次的用户交互上,而脱离于现实世界,从而限制了研究人员的研究思路与研究方法。但这几年,随着移动互联网的迅猛发展,越来越多的用户开始在移动终端使用相应的服务。同时,移动开发者也开发了很多基于地理位置服务(LBS, Location Based Service)的移动应用。这一切使得研究社会网络有了新的方向与思路。移动社交网络(Mobile Social Network)是一种以移动终端为媒介、基于地理位置的社会网络。该网络相对于传统的社交网络更偏重于虚拟社会网络与现实世界之间的交互与联系,



\section{国内外研究现状}






\section{本文主要内容}


\section{本文组织安排}
\begin{itemize}
    \item[1.0] 2012/07/24 已完成大体功能,说明文档和细节方面还有待完善。
    % “a.b”为版本号,b为奇数时为内测版本,为偶数时为发行版。
\end{itemize}
