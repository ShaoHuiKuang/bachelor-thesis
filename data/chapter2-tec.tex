% !Mode:: "TeX:UTF-8"
\chapter{相关理论与技术}
\qquad 移动互联网的迅速发展、社交网络的不断扩大、可研究数据的日益丰富以及机器学习、统计数学、数据挖掘等技术的引入,给社交网络这个领域带来了广泛广泛丰富的研究课题,如信息传播、动态网络演化,网络可视化、Top-K 节点挖掘、社群发现等等。而近些年来关于移动社交网络的研究主要集中在网络的空间性质,物理空间与社交网络的交互以及链接预测。关系识别是对现有网络或者某一个特定时期网络拓扑图中用户之间的关系判别,或者预测等等。从机器学习的角度来看,该问题可以看做是一个分类问题,判断网络中每条边的类型。而分类问题作为机器学习、社交网络中的一个基本问题,一直是该领域研究的热门之一。本文主要介绍移动社交网络的主要特点,并对当前用于该领域的模型及方法进行简要介绍。

\section{社交网络的数据及研究特点}

由于当前研究者掌握研究社交网络的数据各种各样,因此他们的研究也全然不同。有些小组里面的数据多对网络中用户属性的描述,如用户的年龄,性别等,则他们的研究主要在于对用户画像的识别等;另外一些小组的数据如有社交网络整体的变化等数据,则这些小组的研究重点主要在网络的演化等研究领域等。而为了充分利用我们所获取的数据,即社交网络中用户与用户之间的关系类别(如家庭、朋友、同事等),我们的研究重点放在了社交网络中的关系识别上。

从数据分析以及机器学习的角度来看,我们的研究可以转换为一个传统的分类问题(分类出网络中不同的关系类别),许多研究者也的确将这个问题看作分类问题\upcite{cho2011friendship,wang2011human},并且将研究重点放在对移动社交网络的内在特征与特点进行研究上,所以采用的方法大多是基于统计、时间序列的方法,或者传统的机器学习方法,如支持向量机(Support Vector Machine)、决策树(Decision Tree)、罗辑回归(Logistic Regression)等。这些分类方法大多假设数据分布之间是独立同分布的,即每个样本之间并不存在关系(处理时间序列等方法可能会假设服从马尔科夫分布等等),而对于类别的判断主要基于研究者对每个独立样本所提出的特征。而在实际世界中,特别是在移动社交网络拓扑图中,样本之间往往是具有一定的联系的,如Web网页数据、通话数据以及引用数据等等。在通话数据中,每个用户除了拥有自身独特的属性之外,其标签属性还与周围的通话对象有很大的联系。例如,某个用户的联系对象大多为二十岁左右的年轻人,从我们的经验可以基本判断这个人的年龄也大致在二十岁左右。同样在引用网络里面,如果一篇论文的属性属于数据挖掘类,那么它所引用的文章很有可能属于这一大类。这一类数据中节点的标签不仅仅可以从自身的属性上进行推断,也可以从节点所在的拓扑图结构以及周围的信息来进行推断,因为节点与节点之间的信息关联并不是人为假设的,而是在现实世界中自然而然形成的,表示了拓扑图中每个节点之间的相关性与联系。传统方法如上文提到的支持向量机、决策树等,都集中于数据独立同分布,不会采集节点与节点之间的相关信息,这一信息的却是会对关系识别的准确率造成一定的影响。

由于社交网络中数据节点之间的相互关联性,在使用和研究解决社交网路问题的机器学习模型时也需要充分考虑其特性。网络中节点的关联性,往往体现在两点:不同节点之间存在不同的关系,总体的结构非常复杂,需要模型很好的处理这些依赖关系;节点的标签也往往具有关联性,即当我们推断某个A节点时,往往希望能从A节点周围的节点的标签分布并获取一定的信息,从而增加预测A节点标签的准确度。从这两点我们可以看出,我们的模型要么具有很强的联合推断能力,要么能在某些特定的假设情况下,能够忽略这些复杂的依赖关系,并且不会给模型分类的准确度带来影响。这两种模型的思路也代表了两种不同类的模型,在机器学习中,第一类模型常常被称为生成模型(Generative Model),解决问题思路常常从联合概率推断出发,而第二类模型常常被称为判别模型(Discriminative Model),解决问题思路从条件概率推断出发。

\section{概率图模型}