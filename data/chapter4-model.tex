% !Mode:: "TeX:UTF-8"
\chapter{关系识别模型}
\qquad 移动社交网络中的关系识别本身的特点让我们在选取基本模型的时候面对很多挑战:1)移动社交网络中的数据结构以及依赖关系异常复杂,每个变量之间的关系以及结构往往是相互依赖、相互共存的,因此选取的模型需要由很强的表达能力来学习这些依赖,或者模型本身能够通过一定的手段忽略这些依赖关系;2)移动社交网络中的边的标签并不是相互独立而不受影响的,构成边的用户所在的边之间往往是相互联系,之间存在很强的关联性,如何充分利用这种关联性也是我们所需要解决的。

因此,结合第二章我们所介绍的主要几种概率图模型,我们认为我们问题中各类边和点的相互依赖关系难以做出一定的假设来构造马尔科夫网络,故最佳的方式是能够忽略变量之间的关联性,在此基础上充分利用边与团的关联性。因此,我们决定基础模型采用广义的条件虽机场模型,以此基础上构建我们的关系识别模型。由前面第二章的知识我们可以知道,条件虽机场是判别式模型,因此建模的基础事条件概率,这一优势能够有效的忽略变量之间的相互关联性。除此之外,条件虽机场提供了模版团,这正好能够让我们有效利用三元团等特征来提高关系识别的准确率。


模型本身对于各类特征的有效学习之外,我们还希望模型能够在一定程度上减轻标签分布不平衡问题。从我们的数据中统计可以知道,我们的移动社交网络中的关系分布是不均衡的,从整个大的网络分布来看,大约近80\%的关系均为朋友关系,而大约各有10\%左右为家庭关系和同事关系。从前面的知识我们可以知道,概率图模型是基于概率的模型,因此对于标签分布不平衡问题非常敏感。另外,对于我们的问题,这些在总体分布较少的关系往往在实际应用中扮演着至关重要的角色,因此不能说整个关系识别的准确率较高则认为我们的模型较好,而是要具体到这些分布较少的关系识别的准确率当中去。如果其识别的准确率较高,那么我们的模型则较为成功。否则,再高的准确率也说明不了什么问题。因此,在此问题中,我们也必须解决标签分布不平衡问题,并努力提高分布较少的关系类型的识别准确率。



基于前面关系识别特征的发现,我们提出了一个基于条件虽机场(\textbf{Conditional Random Fields})的因子图模型(\textbf{Factor Graph Model}),我们称这个模型为\textbf{BTFG}模型,为Balanced Triadic Factor Graph Model的缩写。在这个模型中,我们从模型角度解决了数据挖掘中常见的数据标签不平衡问题,并且结合了前面的三元团等理论,将团的特性与概率图模型中的模板等概念结合,充分考虑这些团的特征,最大程度上提高关系识别的准确率。


\section{关系识别问题定义}

\begin{problem}
    \label{relation-problem}
     \textbf{关系识别预测}:给定一个一部分被标注的网络$G = (V, E^L, E^U, \bm{X})$,以及一些已经知道的变量$y = 1, 2, 3$ (1代表家庭关系,2代表同事关系,3代表朋友关系),还有一些尚未知道的变量$y = ?$,我们的目标就是要预测这些未知的变量。在我们的模型中,抽象为一个数学问题,则需要学习以下的函数:
     \begin{displaymath}
        {f: G= (V, E^{L}, E^{U}, , Y^{L}),\bm{X} \rightarrow Y^{U}} \\
     \end{displaymath}
     \qquad 用来预测用户之间的关系类型。这里的$Y$即代表了边的关系类型。特别的,$Y^U$代表了那些尚未知道关系类型的边$E^U$。

\end{problem}

在我们研究的工作中,我们将社交关系识别问题看作一个三元分类问题,即将关系类型分为三个群体:朋友,家庭和同事。


下面我们阐述具体模型的推导。


\section{\textit{BTFG} 模型框架}

我们总体的目标是设计一个新颖的模型框架,这个模型框架能够利用和捕捉到上文我们所描述的所有特征,如各类时空特征、三元团特征等等。我们提出了一个能解决标签不平衡、采集三元特征的因子图模型。同时,我们为了解决大规模网络等问题,我们进一步提出了一个高效的分布式学习算法。


\subsection{\textit{Balanced Triadic Factor Graph}}
上文中我们提到采用判别式模型,故我们采用条件概率对我们的问题进行建模。正如前面所说到的一样,采用条件概率进行建模能最大程度忽略不同变量之间的相互依赖关系。在给定用户之间关系的特征和输入移动社交网络的结构条件下,为了使得社交关系变量$Y$的可能性最大,从而我们可以本问题的目标函数。因子图模型提供了一套非常简单、有效的求解全局函数的计算框架,即现可计算不同本地函数的值,然后求各个不同本地函数的乘积的积分,即可得到全局函数。换句话说,即可讲全局函数分解为几个不同本地函数的乘积。因子图模型的这一性质使得最大化目标函数更佳简单,求解更加方便。因此,根据因子图的定义,我们有以下的目标函数:\\
\begin{equation}
P(Y|\textbf{X},G) = \frac{P(\textbf{X},G)P(Y)}{P(\textbf{X},G)} \propto P(\textbf{X}| Y) \times P(Y|G) \\
\end{equation}

其中,$P(Y|G)$代表了给定网络结构下标签的概率,$P(\bm{X} | Y)$ 代表了由属性$\bm{X}$在给定边的标签$Y$的条件下所贡献的概率。下面,我们假设在给定标签的条件下,每条边的属性之间是条件独立的,因此由属性所贡献的概率即可看作每个属性所贡献的概率相乘总的积,因此我们有

\begin{equation}
P(Y|X,G) \propto P(Y|G) \prod_i P(\textbf{x}_i | y_i) \\
\end{equation}


在我们的模型中,我们总共设计了三种不同的因子。第一个因子则是属性因子(Attribute Factor) $f(y_i, x_i)$ ,用来采集两个用户之间的社交关系与其边的基本属性之间的联系。第二个因子是平衡因子$g(y_i)$ ,用于从全局和模型的角度来解决标签分布不平衡问题,具体我们在下面进行阐述。第三个因子是三元结构因子$h(y_c)$ , 用来采集社交关系与我们移动社交网络中三元结构特征之间的关系,这里的$y_c$ 代表了 $y_i, y_j, y_k$ 所组成的集合,如果这三条边能够形成一个封闭的三元三角关系闭元结构。

因此,综上所述,结合我们前面所提到的因子图特性,我们可以进一步降目标函数的联合概率分解为以下的表达形式,

\begin{equation}
\label{eq-sum}
P(Y|G,X) =  \prod_{e_i \in E}f(y_i, \textbf{x}_i) \times \prod_{e_i \in E}g(y_i) \times \prod_{c_{ijk} \in G}h(y_c, \textbf{x}_c)
\end{equation}

从公式中我们可以看出,基于所有随机变量的联合概率分布能够进一步分解为所有局部因子的乘积集合。结合我们的实际社交关系识别问题,我们将三个因子进行的实例化。

\subsection{模型中的三个因子}
从前面的分析中,我们将全局函数分解为三个不同的因子,加下来我们将对这三个因子,结合我们的问题对他们进行实例化。\\

\textbf{属性因子}。我们使用这个因子来代表用户之间的社交关系$y_i$和它的移动网络的社交时空特征$\bm{x}_i$之间的关系。更进一步,我们用以下的线性指数函数来实例化这个因子:

\begin{equation}
\label{eq-attr-factor}
f(y_i, \textbf{x}_i) = \frac{1}{Z_e} \exp(\alpha_i \cdot \textbf{x}_i) \\
\end{equation}

这里的$\alpha$是我们所提出的模型中的的一个参数,而$Z_e$则为标准正则化常量。对于每一条边来说,$\alpha_i$是一个长度为$|\bm{x}|$的向量。而这个参数的第$k$维代表了$x_{ik}$(即$x_i$的第k个属性)对于预测边的标签的贡献程度。比如说,$x_{ik}$代表了两用户之间关系的同现概率,那么这一个因子可以采集不同的社交关系在移动社交网络中所具有的的同现特征。同理,属性因子能够采集其他我们在前面所提到的各种社交时空特征。这部分是所有的概率图模型均会具有的部分,即为我们模型学习的基础,其对社交关系的判别依赖于我们所提出特征的准确性与有效性。\\


\textbf{平衡因子}。 接下来我们定义平衡因子。在定义平衡因子之前,我们现对现有的标签不平衡问题做一个调查回顾。

不平衡问题是数据挖掘领域一个比较经典的问题,曾经在IJCAI和KDD等数据挖掘国际定会上有过专门针对不平衡分类问题的研究主题和讨论。从当前解决不平衡问题的方案来看,主要从数据层面或者算法层面来考虑。

数据层面的方法的目的是通过对数据的采样,改变数据分布从而使原来不平衡的数据分布改变为平衡的数据分布的方法。数据采样的方法又可以分为上采样,下采样和混合采样三种方法。上采样方法又称为过采样方法的目的是增加少数类样本数目,从而改善不平衡分布。下采样方法又称为欠采样方法,其目的是通过减少多数类样本数目的方法使数据分布趋于平衡。两种方法各有优缺点,对于哪种方法更胜一筹也没有严格的证明,于是有研究者将两种方法结合起来提出了混合采样的方法。算法层面的方法主要考虑代价的学习模型,即代价敏感学习方法。此类方法常常对错误分类进行修正,达到对数据集训练重新分分布的目的。除了从代价敏感的角度,最近很多解决方法也boosting 算法来考虑解决。

具体到我们的问题当中,我们希望我们的模型能自己学习数据集中标签不平衡的特性,而不希望破坏网络的整体结构(采用采样的方法必须得劈坏整个网络总体的结构)。因此,我们想能够从第二种方法来考虑,即考虑代价敏感的学习方法,并将此方法加入到我们的模型当中去。结合Yale Song等人\upcite{song2013distribution}应用在隐式条件随机场(\textbf{Hidden Conditional Random Fields})的标签分布敏感参数,我们将其思想推广到我们的因子图模型当中来,这也算是代价敏感学习方法的一种。

这里我们定义平衡因子$g(\bm{y}_i)$, $\bm{y}_i$代表了边$e_i \in E$的社交关系类型。特别的,我们有、

\begin{equation}
\label{eq-balance-factor}
g(y_i) = \frac{1}{Z_n} \exp(\beta_i \cdot \frac{\overline{N}}{N_{y_i}})
\end{equation}

其中我们$\overline{N}$为所有与边$e_i$有公共顶点的边的总数,而$N_y$则为与边$e_i$有着同样标签(社交关系类别)、公共顶点的边的总数。
这样以来,我们的平衡因子即相当于一个标签分布学习因子,能够有效的抑制标签不平衡问题所带来的负面影响。\\


\textbf{三元结构因子}。 最后我们定义三元结构因子来采集社交关系与其社交平衡结构之间的关系。这里,我们有

\begin{equation}
\label{eq-triadic-factor}
h(y_c) = \frac{1}{Z_c}\exp(\sum_{c} \sum_{k}\gamma_c \cdot h_k (\bm{Y}_c)) \\
\end{equation}

对于三元结构因子函数$h_k (\bm{Y}_c))$,我们定义10个特征函数,包含5个平衡结构因子函数以及5个不平衡结构因子函数,如在图\ref{fig-structral-balance}所示。并且这10个函数都被定义为二元函数。更确切的说,如果一个三元结构满足某个二元函数,那么对应的结构因子函数的值为1,否则为0。这一定义参照了概率图模型\upcite{sutton2006introduction},如条件随机场中常用的函数定义方法,简单有效。 \\

最终,我们结合公式\ref{eq-attr-factor}、\ref{eq-balance-factor}、\ref{eq-triadic-factor},将它们带入到公式\ref{eq-sum}中,并将目标函数定义为我们所提出模型的似然,可以得到

\begin{equation}
\begin{split}
\vartheta(\alpha, \beta, \gamma) = \sum_{e_i \in E}\alpha_i \cdot \textbf{x}_i + \sum_{e_i \in E}\beta_i \cdot \frac{\overline{N}}{N_{y_i}}  \\ + \sum_{c_{ijk} \in G} \sum_{k}\gamma_c \cdot h_k (\textbf{Y}_c) - \log{Z}
\end{split}
\end{equation}

这里的$Z = Z_e \cdot Z_n \cdot Z_c$全局标准化变量。

我们所提出的模型能够很好的吸收和消化我们前面所提出的时空以及三元结构特征,并且在训练模型的同时就能解决标签不平衡问题,而不需要破坏网络的整体结构。并且,基于条件概率的判别式模型往往会比基于联合概率的更优,因为我们无需假设某些变量以何种结构依赖于另外一些变量结构,因此我们仅仅需要整体上对所有的变量进行建模,并对社交网络的具体结构进行进一步的分解以适应我们的特征模型,使得模型能够充分利用和学习我们在第三章所提出来的所有类型的特征。这样一来,整个模型的具有一定的实际意义,也是比较容易理解的。

完成对模型的构建构建之后,后续则需要对模型进行求解,以及后续对社交关系的预测。


\section{模型的学习与预测}
现在我们需要来估计参数以及对未知社交关系的用户进行推断。这两个问题其实可以看作一个问题,都可以算作为概率图模型中的推断(\textit{Inference})的过程,如果我们把参数也看作变量的话,即利用一些已知的变量推断另外一些未知的变量。从概率论的角度来看,学习\textit{BTFG}模型就是估计合适的一组参数$ \theta = { \{ \alpha, \beta, \gamma \} } $,来最大化似然概率函数$ \vartheta(\alpha, \beta, \gamma) $。即

\begin{equation}
\theta^\star = \arg \max\vartheta(\theta)
\end{equation}

\subsection{参数学习}
为最优化这个参数学习问题,我们采用一种梯度下降的方法(也被称为 \textbf{Newtown-Raphson}方法)\upcite{luenberger1973introduction}。特别的,我们对每个参数进行求导分解得到,



\begin{equation}
\frac{\partial \vartheta(\theta)}{\partial \alpha}=\bm{E}[\sum_{e_i \in E}\alpha_i \cdot \bm{x}_i] -  \bm{E}_{P_{\alpha}(Y|X)}[\sum_{e_i \in E}\alpha_i \cdot \bm{x}_i]  \\ 
\end{equation}

\begin{equation}
 \frac{\partial \vartheta(\theta)}{\partial \beta}=\bm{E}[\sum_{e_i \in E}\beta_i \frac{\overline{N}}{N_{y_i}}] -  \bm{E}_{P_{\beta}(Y|X,G)}[\sum_{e_i \in E}\beta_i \frac{\overline{N}}{N_{y_i}}] \\
 \end{equation}
\begin{equation}
    \frac{\partial \vartheta(\theta)}{\partial \gamma}=\bm{E}[\sum_{c_{ijk} \in G} \sum_{k}\gamma_c h_k (\bm{Y}_c) ] -  \bm{E}_{P_{\gamma}(Y|X,G)}[\sum_{c_{ijk} \in G} \sum_{k}\gamma_c h_k (\bm{Y}_c) ]  \\
\end{equation}

其中,$\bm{E}[\sum_{e_i \in E}\alpha_i \cdot \bm{x}_i]$ 是在给定的数据$Y$和$\bm{X}$的条件下,属性因子函数求和的期望值。而$\bm{E}_{P_{\alpha}(Y|X)}[\sum_{e_i \in E}\alpha_i \cdot \bm{x}_i]$ 则是给定的参数模型下属性因子函数的期望值。对于参数 $\alpha, \beta$是同样的道理,平衡因子函数与三元结构因子函数在两者条件下的期望值。

由于在我们模型中的图结构是任意的,那么很有可能含有闭环结构。这使得三个公式中的第二项非常难以计算,因为其时间复杂度是对数级别的。为此,们必须采用近似推断的方法来解决这个问题。这里我们采用的是LBP(\textit{Loopy Belief Propagation})\upcite{murphy1999loopy}来计算边缘概率。因LBP容易实现,并且非常的高效,计算方便。

整个参数的学习过程可以被描述为一个迭代算法。在每一次的迭代过程中,都包含两步计算:第一步,我们调用LBP三次,来计算未知变量$P_{\alpha}(Y|X)$, $P_{\theta}(Y|X,G)$ , $P_{\gamma}(Y|X,G)$的边缘分布;第二步,我们使用公式\ref{eq-learning}中的学习更新参数$\eta$来更新$\alpha,\beta, \gamma$。整个学习算法会等到迭代到一定的次数,或者更新参数的幅度过小的时候,会最终停止。


\begin{equation}
\label{eq-learning}
\theta_{new} = \theta_{old} + \eta \cdot \frac{\partial \vartheta(\theta)}{\partial \theta} \\
\end{equation}


\subsection{社交关系预测过程}

有了上述的过程,我们即可以算出估计的参数$\theta$,那么我们的模型即可以确定了。由前面的介绍我们可以知道,其实求参数过程和预测过程基本上都可以算作是一个过程,即利用已知变量推断未知变量,因此我们可以使用刚才求参数类似的思想来预测我们移动社交网络中未知的社交关系${y_i = ?}$,即找到一组社交关系值,使得下面的目标函数的似然最大,

\begin{equation}
Y^{\ast} = \arg \max \vartheta(Y|\textbf{X},G,\theta) \\
\end{equation}

由上述可知,即可同样采用\textit{LBP}来计算边缘概率,最终来估计参数。特别的,我们我们计算每种关系的边缘概率分布$P_(y_i | \textbf{x}_i, G)$ ,最终我们给社交关系赋予那些能够使得最大似然函数的标签。

至此,我们即可以估计出移动社交网络中未知的社交关系。




\section{并行算法实现}
呵呵



































