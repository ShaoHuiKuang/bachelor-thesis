% !Mode:: "TeX:UTF-8"
\chapter*{致谢}
\addcontentsline{toc}{chapter}{致谢}

\qquad 随着成功拿到北美常青藤名校达特茅斯的全奖博士(PhD)录取,我在北航的本科学习告一段落,也即将迎来新的挑战与生活。北航甲子,我有幸成为北航一员,更有幸的成为了北航软件学院1221大班的一员。回望过去四年,太多的人给予我帮助,如今到了毕业之际,也是感恩之时。

首先要感谢的是吕云翔老师。本研究是在我的导师吕云翔老师的指导下完成的。在此之际感谢吕老师在过去四年里对我们悉心指导与关怀。虽然我的毕设是在校外做的,但吕老师仍然严格要求,无论是具体研究分析,还是论文写作,吕老师都给予我细致的指导。吕老师渊博的理论知识、严谨的科研态度以及丰富的学术经验给我留下了深刻的印象,使我受益终生。

感谢我在北大做研究时,指导我的宋国杰老师。当初宋老师不嫌弃我是北大校外学生,能力不够,反而委以重任,手把手教我做研究,让我从事社交网络的研究。是您带我走进科研的世界,让我感受到了科研的魅力。当我选择放弃读您的研究生反而选择出国时,您仍然鼓励我要做自己想做的事情,也继续帮助我从事科研的工作。您对我的这份师生恩情我无以为报,希望您在以后的事业中研究成果愈加丰硕,家庭愈加美满。

感谢大三暑假在卡内基梅隆大学(CMU)做研究时的导师,教授Justine Cassell。您让我切身感受到了真正世界级学术大牛的风采,也让我萌生了以后走学术道路的想法。当然,也感谢您在我申请的时候给予的推荐信,让我有机会在北美常春藤名校直接攻读PhD学位,继续从事自己喜爱的研究工作。同时也感谢博士后Alex和Yoichi的帮助,我有机会在大三暑假做了一些非常有意义的研究工作。

感谢在暑期科研期间所有帮助我的学长和同学们。特别感谢徐可扬学长,作为我的的直系学长,你在学术上的追求以及在为人处事方面一直是我的学习榜样,同时感谢你在我申请和科研路上一直提供帮助,帮我解答人生疑惑,让我坚定了投身科研的信心;感谢实验室的赵冉学长,非常怀念和你在匹兹堡凌晨三四点的时候一起讨论学术问题,希望以后再有机会一起合作;还需要感谢生活上帮助很多的室友胡张柠、秦宇、戴自航学长,希望胡张柠学长在纽约Google工作顺利,秦宇学长顺利毕业找到心仪的工作,戴自航学长在CMU读博一帆风顺,争取能以后和你有Deep Learning方面有意义的合作。


另外感谢在学术生活道路上一起成长的同学刘天毅、张元,祝天毅到了清华后多发文章,研究生申请到好学校,祝张元到北大读博顺利,以后希望能与你一起合作;同时也非常感谢同班的杨缘、李王尔以及耿金坤同学,你们的存在让我明白大神非一日所成;同时需要感谢的我的好朋友李捷,本科最怀念的时光就是和你一起做项目熬夜的时间;另外感谢同班的朱公朴同学,祝你公司能够早日上市;感谢申请路上一直陪伴的北邮杨明同学,清华珊姐和加拿大滑铁卢大学的ViVi,希望你们以后都前程似锦,生活如意。


当然最需要感谢的是始终爱我、关心我、鼓励我的父母。虽然我的父母都没有上过大学,但是在我追逐梦想的道路上,我的父母从来都是支持我的选择,进而我有机会从事计算机科学的研究,走上学术这条道路。


最后需要感谢这一年来参与我答辩的软件学院老师们。在从开题到最终答辩,老师们所提出的宝贵的建议让我获益匪浅,因此我才能最终完成这份研究。为此,谨向软件学院所有的老师致意最诚挚的敬意!


















\cleardoublepage
