% !Mode:: "TeX:UTF-8"
\chapter*{致谢}
\addcontentsline{toc}{chapter}{致谢}

\qquad 四年本科生生涯,如今也到了分别的时刻。从高中毕业到如今,一晃已经四年过去了。北航甲子,我只是这个学校的一个过客。四年的交情并不深厚,我也不是一个重感情的人,但在如今离别之际,竟然也浮生不少伤感。回望过去四年,太多的人给予我帮助,如今到了毕业之际,也是感恩之时。

首先要感谢的是吕云翔老师。本研是在我的导师吕云翔老师的指导下完成的。在此之际感谢吕老师子啊过去四年里对我们悉心指导与关怀。虽然我的毕设是在校外做的,但吕老师仍然严格要求,无论是具体研究分析,还是论文写作,吕老师都给予我细致的指导。吕老师渊博的理论知识、严谨的科研态度以及丰富的学术经验给我留下了深刻的印象,使我受益终生。

感谢我在北大做研究时,指导我的宋国杰老师。当初宋老师不嫌弃我是北大校外学生,能力不够,反而委以重任,手把手教我做研究,让我从事社交网络的研究。是您带我走进科研的世界,让我感受到了科研的美妙。当我选择放弃读您的研究生反而选择出国时,您仍然鼓励我要做自己想做的事情,也继续帮助我从事科研的工作。您对我的这份恩德我无以为报,希望您在以后的事业中研究成果愈加丰硕,家庭愈加美满。

感谢大三暑假在CMU做研究时的导师,教授Justine Cassell。您让我切身感受到了真正世界级学术大牛的风采,也让我萌生了以后走学术道路的想法。当然,也感谢您在我申请的时候给予的推荐信,让我有机会在北美直接读PhD,继续从事自己喜爱的研究工作。同时也感谢在CMU期间给予我非常多帮助的赵冉学长、博士后Alex和Yoichi,使得我有机会在大三暑假做了一些非常有意义的工作。


另外感谢在学术生活道路上一起成长的同学刘天毅、张元,祝天天到了清华后多发paper,早日泡到漂亮妹子,元神到了北大也多发paper,博士期间争取换5个妹子;同时也非常感谢同班的杨缘大大,大帝以及学佛同学,你们的存在让我们5班的成绩直接藐视我们年纪其他班;另外需要感谢的基友李捷,本科最怀念的时光就是和你一起做项目熬夜的时间;另外感谢同班的朱公仆同学,祝你公司能够早日上市,和你妹子感情长久;感谢申请路上一直陪伴的北邮大明,清华三姐和滑铁卢的ViVi,希望你们以后都前程似锦。


最后感谢始终爱我、关心我、鼓励我的父母,特别是我的母亲,没有您的坚定的支持我是无法出国走上学术道路的。

最后,谨向百忙之中抽出宝贵时间来参加论文代答辩的各位专家致以最诚挚的敬意!
















\cleardoublepage
