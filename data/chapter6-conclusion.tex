% !Mode:: "TeX:UTF-8"
\chapter{总结与展望}

\section{完成工作总结}
在本篇论文中,我们通过研究在中国河南省中某个县级市的移动社交网络通话数据,来学习不同社交关系在网络中所展示的通话、时空特征,并在此基础上,我们设计出了一个有效识别社交关系的算法。

在探索不同社交关系所具有的时空特征时,我们发现了一些非常有趣的人与人之间关系的特点。首先,不同社交关系的用户往往在沟通的时间段上各有不同,沟通的时间往往与他们的身份比价相符。第二,不同的社交关系往往会出现在不同类的地方,这侧面反映了他们的日常生活。第三点,不同的社交关系的人群往往在时间和空间上表现各不相同。第四,随着人与人之间交流的深入,人们往往会形成较为稳定的交流圈子。

基于这些发现,我们定义了移动社交网络中关系识别问题,并且提出了一个叫做平衡结构因子图的模型(\textit{Balanced Triadic Factor Graph Model}),并用C++对该算法进行了实现,尽量采用了并行化处理,提高了实现算法的效率。这个模型结合了我们在移动社交网络中各类发现以及标签不平衡问题,并且在关系识别任务上有较好的表现。


我们将我们提出的的模型与当前比较流行的几种算法进行对比,发现我们的模型在移动社交网络中的关系识别任务上能够显著的提高识别准确率。接着,我们又进行了一系列实验来证明我们模型以及我们理论的准确性以及有效性,最后我们可以得出结论,我们的$BTFG$模型在移动社交网络中的关系识别上有着相当高的识别准确度以及稳定性。


\section{未来工作展望}
在移动网络中检测用户之间的社交关系,使得社交网络更加具有真实性,与我们人们生活的世界更加贴切。对于未来的工作,主要有以下几点需要进行:

\begin{itemize}
    \item 首先是地理语义提取的方法需要改进。当前我们采用的方法是$TF-IDF$,该方法较为简单,提取准确率无法得到保障。因此,我们需采用更加有效的方法来提取地理语义。初步的思路是采用LDA\upcite{blei2003latent}来改进地理语义提取。
    \item 当前的时空与社会学理论交叉的不多,但是有很大的空间可以研究。如我们在运用结构不平衡理论的时候,我们其实研究这些稳定的三元团的出行特征,在时空分布与地理去向上面有什么特点。 
    \item 对标签分布不平衡问题解决方法的改进。当前,我们都是从模型的角度来改进标签分布不平衡问题。而在图模型中,我们可以尝试在网络推断的时候进行选择性消息传播,这样能够更加有效的改善标签分布不平衡问题。 
    
    \item 移动社交网络中的分析不仅仅可以识别社交关系,还有许多有趣的应用可以研究。如谣言传播、用户画像刻画、网络推演等等,这些研究在当前社交网络的发展中扮演着越来越重要的角色。因此,下一步我们可以进一步扩展移动社交网络的主题研究。
\end{itemize}




































