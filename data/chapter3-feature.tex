% !Mode:: "TeX:UTF-8"
\chapter{移动社交网络中的关系识别分析}

关系识别是当前社交网络的重要研究课题之一。在一个社交网络中,人们会因为不同的关系而联系在一起,如家人、朋友、同事关系等等。明确社交网络中用户之间不同的关系类型,有利于其它领域的深入研究与发现。如在线或者移动广告营销中,如果知道用户家人、朋友的兴趣爱好以及其常购买的商品类型,那么就能更加准确的给该用户推荐相关的商品与广告,反之亦然,知道用户的喜好,也可以给其朋友、家人等推荐相应合适的商品。在协助公安侦破并抓捕犯罪嫌疑人时,如果能够掌握犯罪嫌疑人其家庭、朋友,则能更快协助相关部门侦破案件,有效的抓捕犯罪人员。由前面介绍可以得知,随着移动智能手机的大规模普及,移动通话数据的人群覆盖率已经接近100\%, 具有相当的普适性。除此之外,移动运营商所提供了家庭套餐、集团套餐等营销套餐,如果研究者能够和移动运营商进行合作,则研究者能够利用从移动运营商中获得的关系数据,作为其训练数据。

从第二章可以得知,从机器学习的角度来看,关系识别实质是一个分类问题。基于目前的研究现状,已经有相当多的学者对此进行了研究。但大多数此类研究都是将关系拆分为简单的“信任与不信任”,“强关系与弱关系”,“友好与敌对”关系,并没有将关系具体到一个明确的网络当中去(如具有家庭、同事、朋友的关系网)。还有部分研究对对关系分类赋予了特定的寓意,但这些研究主要有“指导-被指导(Advisor-Advisee)”\upcite{wang2010mining}、“讲授(Teaching)-指导(Advisor)-助教(Teaching Assistant)”关系\upcite{taskar2003link},比较适用于有向关系,而并非特别适合我们所研究的家庭、同事和朋友关系集。另外一些研究则是基于特别的数据集,如恐怖分子网络数据集分布\upcite{zhao2006entity}, 和我们所要进行研究的通话网络数据结构和性质相差太大,并且这些性质的研究,大多仅从社交层面上对关系进行阐释,而不能从模型的角度充分挖掘社交与空间地理位置之间的联系,而我们所要做的工作则需要从这两个角度同时进行考虑。


移动社交网络提供了非常丰富的信息,可以用来挖掘人们在真实日常生活中的社交关系。在本章,我们首先对基于通话数据中移动社交网络的关系识别问题进行论述和定义,然后将我们所研究的数据进行详细介绍。最后,我们针对从用户通话角度、地理位置同现两个角度进行出发,研究不同交互特征下同事、家庭、朋友关系之间的显著差异,并对特征进行相应的分析。我们用通话数据展示我们的发现。由于篇幅的限制,我们不展示在短信息中的发现,但两者的特征发现比较相近。

\section{关系识别问题定义}
很显然,社交网络是一个图模型,因此不同的问题的基本构成都可以用图$G = (V, E, W)$来进行表示,其中网络图中的每个点$ v_i \in V$表示该网络中的用户,图中点与点的边$Edge(v_i, v_j) \in E$ 表示用户$i$与用户$j$之间存在某种联系(这种联系可以自己定义,如在我们的问题中即两人存在社交关系),而$W$则表示了这种点与点之间的关系强度(如在我们的问题中,则可以定量描述为两用户之间的通话频率与强度等)。

具体到我们的问题当中,我们让$G = (V, E, X, Y)$ 代表无向移动社交网络,这里的$V$ 是$|V| = N$数量的用户集合,而$E \subset V \times V$是表示用户之间社交联系边的集合,每一条边$e_i \in E$ 都有一个相应的社交关系$y_i \in Y$与之对应,这里的$Y \in $\{家庭关系, 同事关系,朋友关系\}。需要注意的是,这里的朋友关系定义为联系较为频繁的用户。$\textbf{X}$是特征矩阵,$x_i$代表了$|x_i|$维特征向量,为每条边$e_i$的特征。因此在解决最终问题,推断移动社交网络前,我们需要选取合适的特征,即$x_i$的值。



\section{数据介绍}
在本论文中所使用的数据集是从2010年10月1日到2010年10月25日采集的中国河南省某县级市的移动手机通话短信数据,包含了30万用户超过六千万(67,630,000)条的通话记录,三千万(31,560,000)条的短信记录,四百万(4,420,000)条的手机开关机纪录,一千二百万条的基站切换纪录。该县级市总共有354座基站,而且每一座基站都有相应的经度和纬度。其中通话、短信的格式如表3.1,开机关机、基站切换的纪录格式如表3.2.




除此之外,我们还有由移动运营商提供的家庭集团和同事工作集团的数据。为了更加精确、更加合理的预测用户之间的关系类别,我们移除了那些家庭集团和工作集团大小为1的孤立点,因为这些点不会对我们所分析的问题构成任何贡献(我们研究的问题本身就是边的关系)。除去这些无用的用户之后,我们可以发现大多数的集团由两个或者三个构成,这类型的集团占了所有家庭和同事集团总数的83\%。并且我们从数据分布上可以发现,同时集团的大小大多小于10人。


\begin{table}
    \centering
    \caption{短信/通话记录格式}
    \label{call-record}
    \begin{tabular}{c|c|c|c|c}
        \hline
        主叫号码 & 被叫号码 & 通话时长 & 主叫基站 & 被叫基站 \\ \hline
        1597128XXXX & 1565295XXXX & 2010-10-20 18:12:34 & 60234 & 60183 \\ \hline
    \end{tabular}
\end{table}


\begin{table}
    \centering
    \caption{事件纪录格式}
    \label{event-record}
    \begin{tabular}{c|c|c|c|c}
    \hline
    事件发生时间 & 用户手机号码 & 时间类型 & 起始基站 & 终止基站 \\ \hline
    2011-10-20 10:10:13   & 135XXXXXXX & 1 & 60284 & 74856  \\ \hline
    \end{tabular}
\end{table}


\begin{figure}[!ht]
    \centering
    \includegraphics[scale=1,width=0.7\textwidth]{figure/FamilyClique.png}
    \caption{家庭集团大小分布}
    \label{fig-familyclique}
\end{figure}

\begin{figure}[!ht]
    \centering
    \includegraphics[scale=1,width=0.7\textwidth]{figure/ColleagueClique.png}
    \caption{同事集团大小分布}
    \label{fig-colleagueclique}
\end{figure}



\section{社交关系识别中的特征分析}

本节主要从移动社交网络中的用户交互、时空交互、社交与时机地理空间交互等角度来分析影响社交关系识别的关键特征,充分利用移动社交网络的社交、空间结合等特性。本小节先从通话社交的角度进行分析,主要分析基本的通话特征对关系识别的影响,以及引入通话熵的概念,然后分析不同关系用户之间时间和通话强度的稳定性与差异性。然后从时空交互的角度来分析,分析了用户时空同现性。随后,分析用户出现地理位置的寓意分析,了解用户同现所在实际位置的具体含义。最后,从经典的社交结构论扩充到地理空间的范围内,提出新的结构洞理论。

\subsection{基本社交通话特征行为分析}

从以前的研究中,我们可以知道,通过用户的通话记录,用户之间的关系(朋友、非朋友)能够















